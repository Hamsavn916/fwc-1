\documentclass{article}
\usepackage{amsmath, amssymb}
\usepackage{graphicx}

\begin{document}

\section*{Exercise 1.3}

\textbf{1.} Let f: \{1,3,4\} \to \{1,2,5\} and g: \{1,2,5\} \to \{1,3\} be given by:
\[
f = \{(1,2), (3,5), (4,1)\}, \quad g = \{(1,3), (2,3), (5,1)\}
\]
Write down g \circ f.

\textbf{2.} Let f, g, h be functions from \mathbb{R} to \mathbb{R}. Show that:
\[
(f + g) \circ h = f \circ h + g \circ h
\]
\[
(f \cdot g) \circ h = (f \circ h) \cdot (g \circ h)
\]

\textbf{3.} Find g \circ f and f \circ g if:
\begin{itemize}
    \item[(i)] f(x) = |x|, g(x) = |5x - 2|
    \item[(ii)] f(x) = 8x^3, g(x) = \frac{1}{x^3}
\end{itemize}

\textbf{4.} If f(x) = \frac{4x+3}{6x-4}, x \neq \frac{2}{3}, show that:
\[
f \circ f (x) = x.
\]

\textbf{5.} Determine whether the following functions have an inverse:
\begin{itemize}
    \item[(i)] f: \{1,2,3,4\} \to \{10\} with f = \{(1,10), (2,10), (3,10), (4,10)\}
    \item[(ii)] g: \{5,6,7,8\} \to \{1,2,3,4\} with g = \{(5,4), (6,3), (7,4), (8,2)\}
    \item[(iii)] h: \{2,3,4,5\} \to \{7,9,11,13\} with h = \{(2,7), (3,9), (4,11), (5,13)\}
\end{itemize}

\textbf{6.} Show that f: [-1,1] \to \mathbb{R}, given by f(x) = \frac{x}{x+2}, is one-one. Find the inverse of the function f: [-1,1] \to \text{Range } f.  
(Hint: For y \in \text{Range } f, solve y = \frac{x}{x+2} for x, i.e., x = \frac{2y}{1-y}).

\textbf{7.} Consider f: \mathbb{R} \to \mathbb{R} given by f(x) = 4x + 3. Show that f is invertible. Find the inverse of f.

\textbf{8.} Consider f: \mathbb{R}_+ \to [4, \infty) given by f(x) = x^2 + 4. Show that f is invertible with the inverse:
\[
f^{-1}(y) = \sqrt{y-4},
\]
where \mathbb{R}_+ is the set of all non-negative real numbers.

\newpage

\textbf{9.} Consider f: \mathbb{R}_+ \to [-5, \infty) given by f(x) = 9x^2 + 6x - 5. Show that f is invertible with:
\[
f^{-1}(y) = \frac{\sqrt{y+6} -1}{3}.
\]

\textbf{10.} Let f: X \to Y be an invertible function. Show that f has a unique inverse.  
(Hint: Suppose g_1 and g_2 are two inverses of f. Then for all y \in Y,  
\[
f \circ g_1(y) = 1_Y(y) = f \circ g_2(y).
\]
Use the one-one property of f.)

\textbf{11.} Consider f: \{1,2,3\} \to \{a,b,c\} given by:
\[
f(1) = a, \quad f(2) = b, \quad f(3) = c.
\]
Find f^{-1} and show that (f^{-1})^{-1} = f.

\textbf{12.} Let f: X \to Y be an invertible function. Show that the inverse of f^{-1} is f, i.e.,
\[
(f^{-1})^{-1} = f.
\]

\textbf{13.} If f: \mathbb{R} \to \mathbb{R} is given by f(x) = (3 - x^3)^{1/3}, then f \circ f(x) is:
\[
\text{(A) } \frac{1}{x^3}, \quad
\text{(B) } x^3, \quad
\text{(C) } x, \quad
\text{(D) } (3 - x^3).
\]

\textbf{14.} Let f: \mathbb{R} - \left\{ -\frac{4}{3} \right\} \to \mathbb{R} be a function defined as:
\[
f(x) = \frac{4x}{3x+4}.
\]
The inverse of f is the map g: \text{Range } f \to \mathbb{R} - \left\{ -\frac{4}{3} \right\} given by:
\[
\text{(A) } g(y) = \frac{3y}{3 - 4y}, \quad
\text{(B) } g(y) = \frac{4y}{4 - 3y}, \quad
\text{(C) } g(y) = \frac{4y}{3 - 4y}, \quad
\text{(D) } g(y) = \frac{3y}{4 - 3y}.
\]

\end{document}

